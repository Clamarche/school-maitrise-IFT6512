\documentclass{beamer}

\usepackage[utf8]{inputenc}
\usetheme{Singapore}
\usepackage{xcolor}
\setbeamertemplate{footline}[frame number]

%\usepackage{mathlist}

\usepackage[utf8]{inputenc}
\usepackage{amsmath,amssymb}

\newtheorem{defi}{Définition}
\newtheorem{theo}{Théorème}
\newtheorem{coro}{Corollaire}
\newtheorem{lem}{Lemme}

%\usepackage{epic,eepic}
%\usepackage{pstricks, pst-tree}

\usepackage{times}

\usepackage{ulem}
%\usepackage{crayola}

\usepackage{listings}
%\topmargin=-0.6in
\lstloadlanguages{C++}
\lstset{language=C++}
%%}

%\usepackage{gnuplottex}
% This variables allows to conditional generate the Gnuplot figures.
%\def \gnuplotx {Generate the figures}

\usepackage{graphicx}
\usepackage{epstopdf}

\setbeamercovered{dynamic}

\def\bxi{\boldsymbol\xi}
\def\bepsilon{\boldsymbol\epsilon}
\def\bomega{\boldsymbol\xi}
\def\aff{\operatorname{aff}}
\def\cR{\mathcal{R}}

\newcommand{\tim}[1]{\;\; \mbox{#1} \;\;}

\title[CP]{Stochastic optimization\\Chance constrained programming}

\author{Fabian Bastin\\\url{fabian.bastin@polymtl.ca}}

\date{}

\begin{document}

%% AJOUTER DISCUSSION BIRGE ET LOUVEAUX P128

\frame{\titlepage}

\begin{frame}
\frametitle{Motivation}

We consider the toy problem (taken from J. Linderoth)
\begin{align*}
\min_x\ & x_1 + x_2 \\
\mbox{s.t. } & \xi_1 x_1 + x_2 \geq 7 \\
& \xi_2 x_1 + x_2 \geq 4 \\
& x_1, x_2 \geq 0,
\end{align*}
where $\xi_1 \sim U(1,4)$, $\xi_1 \sim U(1/3,1)$.

\mbox{} 

Instead of requiring that a constraint holds for all the scenarios, we can require a sufficiently large probability to satisfy a constraint.

\end{frame}

\begin{frame}
\frametitle{Chance constraints}

\begin{enumerate}
\item 
Separate chance constraints
\begin{align*}
P [ \xi_1x_1 + x_2 \geq 7 ] &\geq \alpha_1 \\
P [ \xi_2x_1 + x_2 \geq 4 ] &\geq \alpha_2
\end{align*}
\item
Joint (integrated) chance constraint
\[
P [ \xi_1x_1 + x_2 \geq 7 \cap \xi_2x_1 + x_2 \geq 4 ] \geq \alpha
\]
\end{enumerate}

\end{frame}

\begin{frame}
\frametitle{Example: joint chance constraints}

\begin{align}
P[(\xi_1, \xi_2) = (1,1)] &= 0.1 \\
P[(\xi_1, \xi_2) = (2,5/9)] &= 0.4 \\
P[(\xi_1, \xi_2) = (3,7/9)] &= 0.4 \\
P[(\xi_1, \xi_2) = (4,1/3)] &= 0.1
\end{align}

\mbox{}

Assume that $\alpha \in (0.8,0.9]$, and we have the joint constraint
\[
P [ \xi_1x_1 + x_2 \geq 7 \cap \xi_2x_1 + x_2 \geq 4 ] \geq \alpha
\]

\mbox{}

We then have to satisfy constraints (2) and (3) and either (1) or (4).

\end{frame}

\begin{frame}
\frametitle{Example: graph}

\begin{center}
\includegraphics<1>[width=\textwidth,keepaspectratio]{graph.eps}
\end{center}

\end{frame}

\begin{frame}
\frametitle{Properties}

Feasible set
\[
K_1(\alpha) = \lbrace x \,|\, P[T(\xi)x \geq h(\xi)] \geq \alpha \rbrace
\]
$K_1(\alpha)$ is not necessarily convex.

\mbox{}

\begin{theorem}
Suppose $T(\xi) = T$ is fixed, and $h(\xi)$ has a quasi-concave
probability measure $P$. Then $K_1(\alpha)$ is convex for $0 \leq \alpha \leq 1$.
\end{theorem}

\mbox{}

A function $P: D \rightarrow \mathcal{R}$ defined on a domain $D$ is quasi-concave if $\forall$ convex sets $U, V \subseteq D$, and $0 \leq \lambda \leq 1$,
\[
P[(1-\lambda)U+\lambda V] \geq \min \lbrace P[U], P[V] \rbrace.
\]

\end{frame}

\begin{frame}
\frametitle{Quasi-concave probability distributions}

\begin{itemize}
\item 
Uniform
\[
f(x) =
\begin{cases}
1/\mu(S), & x \in S \\
0 & \mbox{otherwise},
\end{cases}
\]
where $\mu(S)$ is the measure of $S$.
\item 
Exponential density
\[
f(x) = \lambda e^{-\lambda x}
\]
\item 
Multivariate normal density:
\[
f(x) = \frac{1}{\sqrt{(2\pi)^n/2\det(\Sigma)}}e^{-\frac{1}{2}(x-\mu)'\Sigma (x-\mu)}
\]
\end{itemize}

\mbox{}

If you have such a density, you can
\begin{itemize}
\item
use Lagrangian techniques
\item
use a reduced-gradient technique (see Kall \& Wallace, Section~4.1)
\end{itemize}

\end{frame}

\begin{frame}
\frametitle{Single constraint: easy case}

The situation in the single constraint case is somewhat more simple.

\mbox{}

Suppose again that $T_i (\xi) = T_i$ is constant. Then
\[
P[T_i x \geq h_i (\xi)] = F(T_i x) \geq \alpha
\]
so the deterministic equivalent is
\[
T_i x \geq F^{-1}(\alpha)
\]
\ldots linear constraint! The resulting problem is still linear.

\mbox{}

Recall that the inverse of the cdf is defined as
\[
F^{-1}(\alpha) = \min \lbrace x : F (x)  \geq \alpha \rbrace.
\]

\end{frame}

\begin{frame}
\frametitle{Other ``solvable'' cases}

Let $h(\xi) = h$ be fixed, $T(\xi) = (\xi_1, \xi_2,\ldots, \xi_n)$, with $\xi = (\xi_1, \xi_2,\ldots, \xi_n)$ a multivariate normal distribution with mean $\mu = (\mu_1, \mu_2,\ldots,\mu_n)$ and variance-covariance matrix $\Sigma$.
Then
\[
K_1(\alpha) = \lbrace x \,|\, \mu' x \geq h + \Phi^{-1}(\alpha) \sqrt{ x' \Sigma x } \rbrace,
\]
where $\Phi$ is the standard normal cdf.

\mbox{}

$K_1(\alpha)$ is a convex set for $\alpha \geq 0.5$.

\mbox{}

It is possible to express it as a second order cone constraint:
$$
\| \Sigma^{1/2} x \|_2 \leq \frac{1}{\Phi^{-1}(\alpha)} (\mu' x - h)
$$

\end{frame}

\begin{frame}
\frametitle{Second-order cone programming}

A second-order cone program (SOCP) is a convex optimization problem of the form
\begin{align*}
\min_x \ & f^T x \\
\mbox{s.t. } &
\| A_i x + b_i \|_2 \leq c_i^T x + d_i,\ i = 1,\ldots,m \\
& F x = g
\end{align*}
where $x \in \cR^n$, $f, c_i \in \cR^n$, $A_i \in \cR^{n_i \times n}$, $b_i \in \cR^{n_i}$, $d_i \in \cR$, $F \in \cR^{p \times n}$, and $g \in \cR^p$.

\mbox{}

SOCPs can be solved by interior point methods.

\end{frame}

\begin{frame}
\frametitle{Example: robust portfolio optimization}

(Taken from S. Boyd and J. Linderoth)

Suppose we want to invest in $n$ assets, providing return rates $\beta_1, \beta_2,\ldots, \beta_n$.

\mbox{}

The $\beta_i$'s are random variables.
Assume that they are following a multivariate normal distribution with means $\beta_i$ and covariance matrix $\Sigma$.

\mbox{}

Suppose that we want to ensure a return of at least $T$.
We cannot guarantee it all the time, but we want it to occur most of the time.

\end{frame}

\begin{frame}
\frametitle{Example: robust portfolio optimization (cont'd)}

Let $x_i \geq 0$ the part of portfolio to invest in stock $i$.
We have the constraints
\begin{align*}
& P\left[ \sum_{i = 1}^n \beta_i x_i \geq T  \right] \geq \alpha  \\
& \sum_{i = 1}^n x_i \leq x \\
& x_i \geq 0,\ i = 1,\ldots,n
\end{align*}
where $x$ is the total amount to invest.

\mbox{}

The chance constraint can be rewritten as
\[
\beta'x - \Phi^{-1}(\alpha) \sqrt{x'\Sigma x} \geq T.
\]


\end{frame}

\begin{frame}
\frametitle{Example: robust portfolio optimization (cont'd)}

We can also interpret $x_i$ as proportion of the portfolio (position of asset $i$), by normalizing $\| x \|_1$ to 1.
$T$ is now the minimum return rate of the portfolio and $x$ is the portfolio allocation.

\mbox{}

We can add some constraints on the $x_i$ to ensure diversification. We summarize them by requiring $x \in \mathcal{C}$.

\mbox{}

A complete program can now be expressed as
\begin{align*}
\max_x \ & E[\beta'x] \\
\mbox{s.t. } & P\left[ \beta' x \geq T \right] \geq \alpha  \\
& \sum_{i = 1}^n x_i = 1 \\
& x \in \mathcal{C}
\end{align*}

\end{frame}

\begin{frame}
\frametitle{Example: loss constraint}

Setting $T$ to 0 means that we want to ensure that we will no suffer from loss with some probability. Typicially, $\alpha$ is set to 0.9, 0.95, 0.99,\ldots

\mbox{}

The chanced-constraint can also be expressed as
\[
P\left[ \beta' x \leq 0 \right] \leq 1-\alpha = \gamma.
\]

\mbox{}

We can also allow the sale of some parts of the portfolio by allowing some $x_i$ to be negative.
\end{frame}

\begin{frame}
\frametitle{Numerical illustration}

(Taken from S. Boyd -- \url{http://ee364a.stanford.edu/lectures/chance_constr.pdf})

$n = 10$ assets, $\alpha = 0.95$, $\beta = 0.05$, $\mathcal{C} = \{x | x \succeq -0.1\}$

\mbox{}

Compare
\begin{itemize}
\item
optimal portfolio
\item
optimal portfolio without loss risk constraint
\item
uniform portfolio $(1/n)\mathbf{1}$
\end{itemize}

\mbox{}

\begin{center}
\begin{tabular}{c|c|c}
portfolio & $E[\beta'x]$ & $P[\beta' x \leq 0])$ \\
\hline
optimal & 7.51 & 5.0\% \\
w/o loss constraint & 10.66 & 20.3\% \\
uniform & 3.41 & 18.9\%
\end{tabular}
\end{center}

\end{frame}

\begin{frame}
\frametitle{Other situations}

Usually very hard.

\mbox{}

Use a bounding approximation or sample average approximation (SAA).

\end{frame}

\end{document}