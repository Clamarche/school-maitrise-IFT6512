\documentclass{beamer}

\usetheme{Warsaw}

%\usepackage{mathlist}

\usepackage[utf8]{inputenc}
\usepackage{amsmath,amssymb,amsthm}

\newtheorem{defi}{Définition}
\newtheorem{theo}{Theorem}
\newtheorem{coro}{Corollaire}
\newtheorem{lem}{Lemme}

%\usepackage{epic,eepic}
%\usepackage{pstricks, pst-tree}

\usepackage{times}

\usepackage{ulem}
%\usepackage{crayola}

\usepackage{listings}
%\topmargin=-0.6in
\lstloadlanguages{C++}
\lstset{language=C++}
%%}

%\usepackage{gnuplottex}
% This variables allows to conditional generate the Gnuplot figures.
%\def \gnuplotx {Generate the figures}

\usepackage{graphicx}
\usepackage{epstopdf}

\setbeamercovered{dynamic}

\def\bxi{\boldsymbol\xi}
\def\bepsilon{\boldsymbol\epsilon}
\def\bomega{\boldsymbol\xi}
\def\aff{\operatorname{aff}}

\newcommand{\tim}[1]{\;\; \mbox{#1} \;\;}

\title[CP]{Stochastic optimization\\Chance constrained programming - SAA}

\author[Fabian Bastin]{Fabian Bastin\\\url{fabian.bastin@cirrelt.ca}\\Université de Montréal -- CIRRELT}

\date{}

\setbeamertemplate{footline}[frame number]

\begin{document}

\frame{\titlepage}

\begin{frame}
\frametitle{Motivation}

We consider the problem
\begin{align*}
\min_{x \in \mathcal{X}}\ & f(x) \\
\mbox{s.t. } & P[G(x,\xi) \leq 0] \geq 1 - \alpha,
\end{align*}
where $G: \mathcal{R}^n \times \Xi \rightarrow \mathcal{R}^m$.

This allows to consider separate and joint chanced constrained.

We can reformulate this problem as
\begin{align*}
\min_{x \in \mathcal{X}}\ & f(x) \\
\mbox{s.t. } & P[G(x,\xi) > 0] \leq \alpha,
\end{align*}


\mbox{}

If $P[G(x,\xi) \leq 0]$ has no closed form, we can always approximate it by sample average approximation.

\end{frame}

\begin{frame}
\frametitle{SAA}

With the indicator function of $(0,\infty)$
\[
\mathcal{I}_{(0,\infty)} =
\begin{cases}
1 & \mbox{if } t > 0,\\
0 & \mbox{if } t \leq 0,
\end{cases}
\]
we can rewrite the chance constraint as
\[
p(x) = E\left[ \mathcal{I}_{(0,\infty)} \left( G(x, \xi) \right) \right] \leq \alpha.
\]

Let $\xi_1,\ldots,\xi_N$ be an independent identically distributed (iid) sample of $N$ realizations of $\xi$.
We approximate the chance constraint by
\[
\hat{p}_N = \frac{1}{N} \sum_{j = 1}^N \mathcal{I}_{(0,\infty)} \left( G(x, \xi_j) \right) \leq \gamma
\]

\end{frame}

\begin{frame}
\frametitle{SAA (cont'd)}

$\hat{p}_N(x)$ is equal to the proportion of times that $G(x,\xi_j) > 0$.
$\gamma$ is not necessarily equal to $\alpha$, especially when we want to enlarge the feasible of the approximate problem.

\mbox{}

The approximate problem is
\begin{align*}
\min_{x \in \mathcal{X}}\ & f(x) \\
\mbox{s.t. } & \hat{p}_N(x) \leq \gamma.
\end{align*}

\end{frame}

\begin{frame}
\frametitle{Theory}

For theoretical justifications, see
\begin{quote}
B.K. Pagnoncelli, S. Ahmed, and A. Shapiro, ``Sample Average Approximation Method for Chance
Constrained Programming: Theory and Applications'', Journal of Optimization Theory and Applications 142 (2), pp. 399--416, 2009.
\end{quote}

\mbox{}

Especially, under some regularity conditions, we have
\[
sup_{x \in C} |\hat{p}_N(x) - p(x)| \rightarrow 0,
\]
w.p.1, as $N \rightarrow \infty$.

\end{frame}

\begin{frame}
\frametitle{Convergence}

Assume that  there is an optimal solution $\overline{x}$ of the true problem such that for any $\epsilon > 0$ there is $x \in \mathcal{X}$ with $\| x - \overline{x} \| \leq \epsilon$ and $p(x) < \alpha$.

\mbox{}

\begin{theorem}
Suppose $\gamma = \alpha$, $\mathcal{X}$ compact, and $f(x)$ continuous.
Then $\hat{v}_N \rightarrow v^*$, where $\hat{v}_N$ and $v^*$ are the optimal values of the approximate and true problems, respectively, and $D(\hat{S}_N,S) \rightarrow 0$ almost surely as $N \rightarrow \infty$.
\end{theorem}

\end{frame}

\begin{frame}
\frametitle{Application: call centers}

Work in progress with Thuy Anh Ta and Pierre L'Ecuyer.
The next slides are due Thuy Anh Ta.

\mbox{}

{\color{red}A call center}: an  office used for receiving or transmitting customers requests by {\color{red}telephone}.


	\begin{itemize}
		\item A {\color{blue}call} represents a customer and it is classified by {\color{blue}type} ($K$ types). 
		\item Customers are served by {\color{blue}agents}.
		\item Agents sharing a common set of tasks form a {\color{blue}group} of agents ($I$ groups).
		\item A day is divided into {\color{blue}periods} ($P$ periods).
		\item The work schedule of an agent is determined by her {\color{blue}shift} ($Q$ shift configurations).
\end{itemize}
	
\end{frame}

\begin{frame}{Performance measure}
	\begin{itemize}
		\item{} {\color{blue}{Performance measures} (PM)} allow to assess the quality of service (QoS) and
		efficiency of a call center, e.g., Service level (SL) and average waiting time (AWT), abandonment ratio, occupancy ratio, etc.
		\item {\color{blue}Service level (SL)}: the percentage of calls that are answered in a defined waiting threshold.
		\item {\color{blue}Average waiting time (AWT)}: the average (or mean) time a customer waited to have a service.
		\item{} QoS can be defined in the long run (expectations), or over a given time period (random variable).
	\end{itemize}
\end{frame}

\begin{frame}{Staffing, Scheduling and Routing}
	\begin{itemize}
		\item{} The objective is to minimize the operating cost of a call center under a set of constraints on certain PMs (SL or AWT).
		\item{} Staffing: deciding the number of agents $y_{i,p}$ of each group for each period in a day.
		\item{} Scheduling:  determining how many agents $x_{i,q}$ to assign to a set of predefined shifts.
		\item{} Routing: determining which agents are allowed to handle the call, and how agents are chosen when several agents are free.
		
	\end{itemize}
	%		\begin{figure}
	%			\includegraphics[width=0.4\linewidth]{figures/planning.pdf}%
	%		\end{figure} 
\end{frame}

\begin{frame}{Modeling a call center}
	\begin{itemize}
		%		\item{} Different types of calls arrive at random and different groups of agents answer these calls
		\item{} The calls arrive according	to arbitrary stochastic processes that could be non-stationary (e.g. homogeneous Poisson process), and perhaps doubly stochastic (e.g. Poisson gamma process, the arrival rate is a random	variable of gamma distribution).
		\item{} Service times: exponential, log-normal distributions.
		%of calls are traditionally assumed as independent and identically distributed exponential random variables with a constant mean.
		\item{} Delay time,  patience time, etc.
		\item{} When number of skills is low: Arrivals are served in a first come first serve (FCFS) order and/or a longest idle server first rules. 
		\item{} Otherwise, {\color{blue}skills based routing} may be used to route calls to appropriate agents.
	\end{itemize}
\end{frame}

\begin{frame}{Staffing and scheduling with predictable arrival rates}
	\[
	g(y) = \frac{\mathbb{E}[\# \text{of served call that waited at most } \tau]}{\mathbb{E}[\# \text{ Total } \# \text{ of calls}]}
	\]
	\textbf{Scheduling problem}	
	\[
	\min {c}^T{x} = \sum_{i=1}^I\sum_{q=1}^Q c_{i,q}x_{i,q}
	\]
	subject to:
	\begin{equation} \tag{P1}
	\begin{aligned}
	{Ax} &\geq {y} \\
	g_{k,p}({y}) &\geq l_{k,p} ~\forall k,p, \\
	g_{p}({y}) &\geq l_{p} ~\forall p, \\
	g_{k}({y}) &\geq l_{k} ~\forall k, \\
	g(y) &\geq l,\\
	{x},{y}&\geq 0 \text{ and integer}  \text{.}\\
	\end{aligned}
	\end{equation}
\end{frame}

\begin{frame}{Staffing and scheduling with predictable arrival rates}
	\textbf{Staffing problem}	
	\[
	\min_y {c'}^T y = \sum_{i=1}^I\sum_{p=1}^P c'_{i,p}y_{i,p}
	\]
	subject to:
	\begin{equation} \tag{P2}
	\begin{aligned}
	g_{k,p}({y}) &\geq l_{k,p} ~\forall k,p, \\
	g_{p}({y}) &\geq l_{p} ~\forall p, \\
	g_{k}({y}) &\geq l_{k} ~\forall k, \\
	g(y) &\geq l,\\
	{y}&\geq 0 \text{ and integer}  \text{.}\\
	\end{aligned}
	\end{equation}
\end{frame}

\begin{frame}
	
	SL $S(\tau, y)$ and AWT $W(y)$ are considered as random variables in a given period.
	
	The chance-constrained scheduling problem for multiskill call centers is formulated as:
	\[
	\min {c}^T{x} = \sum_{i=1}^I \sum_{q=1}^Q c_{i,q}x_{i,q}
	\]
	subject to:
	\begin{equation} \tag{P3}
	\begin{aligned}
	{Ax} &\geq {y} \\
	\mathbb{P}[S_{k,p}(\tau_{k,p}, {y}) \geq s_{k,p}] &\geq r_{k,p} \text{\qquad{}}  \forall k,p \\
	\mathbb{P}[W_{k,p}(y) \leq w_{k,p}] &\geq  v_{k,p} \text{\qquad{}} \forall k,p\\
	{x}, {y} &\geq 0 \text{ and integer}. \\
	\end{aligned}
	\end{equation}
\end{frame}


\begin{frame}{Sample average approximate problem}
	\[
	\min {c}^T{x} = \sum_{i=1}^I \sum_{q=1}^Q c_{i,q}x_{i,q}
	\]
	subject to:
	\begin{equation} \tag{SAA} \label{s1}
	\begin{array}{ll}
	{Ax} &\geq {y} \\
	\displaystyle \frac{1}{N} \sum_{d=1}^N\mathbb{I}[\hat{S}_{N, k,p}^d(\tau_{k,p}, {y}) \geq s_{k,p}] &\geq r_{k,p} \text{\qquad{}}\forall k,p\\
	\displaystyle \frac{1}{N}\sum_{d=1}^N\mathbb{I}[\hat{W}_{N,k,p}^d({y}) \leq w_{k,p}]&\geq v_{k,p} \text{\qquad{}}\forall k,p\\
	{x}, {y} &\geq 0 \text{ and integer}  \text{ .}\\
	\end{array}
	\end{equation}
	Under some specific conditions, the optimal value and the set of optimal solutions of the SAA problem converge to those of the true problem  with probability one as $N$ approaches infinity. 
\end{frame}

\end{document}