\documentclass{beamer}
%\documentclass[xcolor=pst,dvips,epic,eepic]{beamer}

\usepackage[utf8]{inputenc}
\usetheme{Singapore}
\usepackage{xcolor}
\setbeamertemplate{footline}[frame number]

%\usepackage{pgf,pgfarrows,pgfnodes,pgfautomata,pgfheaps,pgfshade}
%\usepackage[pst]{xcolor}
%\usepackage{xxcolor}
\usepackage{mathlist}

\usepackage[utf8]{inputenc}
\usepackage{amsmath,amssymb}
\usepackage{colortbl}
%\usepackage[english]{babel}
%\usepackage{pstricks}
%\usepackage{pst-tree}
%\usepackage{pictex}
\usepackage{tikz}
\usetikzlibrary{trees}
\usepackage{tikz-qtree}

\usepackage{times}

%\usepackage{pst-tree}

\usepackage{ulem}

\usepackage{listings}
%\topmargin=-0.6in
\lstloadlanguages{C++}
\lstset{language=C++}
%%}

\setbeamercovered{dynamic}

\def\bxi{\boldsymbol\xi}
\def\bomega{\boldsymbol\omega}

\def\eqdef {\buildrel \rm def \over =}    % Egal par definition.
\def\eqas  {\buildrel \rm a.s. \over =}   % Egal a.s.
\def\toas  {\buildrel \rm a.s. \over \to} % ---> a.s.
\def\q     {$\kern1.4em$}    % Indentation spaces for slides, algor, programs.
\def\?{\discretionary{}{}{}}  % Same as \- but does not print the - sign
\def\g   {{\;\leftarrow\;}}

\def\E {{\mathbb E}}
\def\EE {{\mathbb E}}
\def\FF {{\mathbb F}}
\def\II {{\mathbb I}}
\def\LL {{\mathbb L}}
\def\P {{\mathbb P}}
\def\PP {{\mathbb P}}
\def\QQ {{\mathbb Q}}
\def\RR {{\mathbb R}}
\def\ZZ {{\mathbb Z}}

\def\Var{{\rm Var}}
\def\Cov{{\rm Cov}}
\def\MSE{{\rm MSE}}
\def\RE{{\rm RE}}
\def\REff{{\rm REff}}
\def\Eff{{\rm Eff}}
\def\eff{{\rm Eff}}
\def\tr{{\sf T}}
\def\byC {{\bf \yel{C}}}
\def\byh {{\bf \yel{h}}}
\def\byq {{\bf \yel{q}}}
\def\byY {{\bf \yel{Y}}}
\def\byzero {{\bf \yel{0}}}
\def\bySigma {{\bf \yel{\Sigma}}}

\def\cF {{\mathcal{F}}}
\def\cQ {{\mathcal{Q}}}
\def\cR {{\mathcal{F}}}

\def\blue{\color{blue}}
\def\red{\color{red}}

\title[Stochastic optimization]{Stochastic optimization}

\author[Fabian Bastin]{Fabian Bastin \\ \url{fabian.bastin@cirrelt.ca} \\ Université de Montréal -- CIRRELT}

\date{}

\begin{document}

\frame{\titlepage}

\begin{frame}
\frametitle{Introduction}

Consider the general deterministic program
\[
\begin{split}
& \min g_0(x) \\
& \text{s.t. }g_i(x) \leq 0, i = 1,\ldots{},m \\
& \phantom{t.q. }x \in X \subset \cR^n.
\end{split}
\]

\mbox{}

All the parameters are assumed to be perfectly known. {\color{red} Realistic?}
\begin{itemize}
\item
measurement errors;
\item
uncertainties on the future;
\item
data unavailable;
\item
\ldots
\end{itemize}

\end{frame}

\begin{frame}
\frametitle{Mathematical programming and stochastic programming}

\begin{itemize}
\item
{\red Mathematical programming} (optimization): typically: decision problem (where the meaning of the term ``decision'' is broad).
\item
{\red Stochastic programming} concerns decision under uncertainty, the uncertainty being represented by means of random parameters.
\end{itemize}
\[
\begin{split}
& ``\min_{x \in X}" g_0(x,\bxi) \\
& \text{s.t. }g_i(x, \bxi) \leq 0,\ i=1,\ldots{},m,\\
\end{split}
\]
where $\bxi$ is a random vector. Meaning of ``$\min$''?

\mbox{}

{\red Assumption}: we can represent the uncertainty by means of the (joint) probability distribution.

\end{frame}

\begin{frame}
\frametitle{The farmer problem}

Source: Birge et Louveaux, Section~1.1.

\mbox{}

{\blue Scenarios approach}
\begin{itemize}
\item
One assumes that the random vector is finite. Each of the realization is a scenario.
\item
Even if the random vector is continuous, a discrete approximation is often useful.
\end{itemize}

\mbox{}

A European farmer has 500 acres of land and cultivates wheat, corn and sugar beets.

\mbox{}

At least 200T of wheat and 240T of corn are needed to feed his livestock.
Any additional production can be sold, but in case of underproduction, he has to buy the complement, with a purchase cost 40\% greater than the sale cost.
The farmer can sold the sugar beets at \$36T for the first 6000 tons, and \$10T after, due to European quotas.

\end{frame}

\begin{frame}
\frametitle{The farmer problem II}

\begin{center}
\begin{tabular}{lccc}
\hline
Culture & Wheat & Corn & Sugar beets \\
\hline
Average return ($T$) & 2.5 & 3 & 20 \\
Plantation cost (\$/acre) & 150 & 230 & 260 \\
Selling price (\$/T) & 170 & 150 & 36 ($\leq$ 6000T), 10 \\
Buying price (\$/T) & 238 & 210 & - \\
Minimum required (T) & 200 & 240 & - \\
\hline
\end{tabular}
\end{center}

\mbox{}

Notations:
\begin{itemize}
\item
$x_1$, $x_2$, $x_3$: acres for wheat, corn, sugar beets;
\item
$y_1$, $y_2$: tons of bought wheat and corn;
\item
$w_1$, $w_2$: tons of sold wheat and corn;
\item
$w_3$, $w_4$: tons of sold sugar beets, at high price and at low price.
\end{itemize}

\mbox{}

How to decide the surface to allocate to each plant?

\end{frame}

\begin{frame}
\frametitle{The farmer problem: deterministic version}

Linear program:
\begin{align*}
\min\ & 150x_1 + 230x_2 + 260x_3 +\\
&  238y_1 - 170w_1 + 210y_2 - 150w_2 -36w_3 - 10w_4 \\
\text{t.q. } & x_1 + x_2 + x_3 \leq 500; \\
& 2.5x_1 + y_1 - w_1 \geq 200; \\
& 3x_2 + y_2 - w_2 \geq 240; \\
& w_3 + w_4 \leq 20 x_3; \\
& w_3 \leq 6000; \\
& x_1, x_2, x_3, y_1, y_2, w_1, w_2, w_3, w_4 \geq 0.
\end{align*}

\end{frame}

\begin{frame}
\frametitle{The farmer problem: deterministic solution}

Total (expected) profit: \$118600. Details:
\begin{center}
\begin{tabular}{lccc}
\hline
Culture & Wheat & Corn & Sugar beets \\
\hline
Surface (acres) & 120 & 80 & 300 \\
Production (T) & 300 & 240 & 6000 \\
Sales (T) & 100 & - & 6000 \\
Purchase (T) & - & - & - \\
\hline
\end{tabular}
\end{center}

\mbox{}

The production is however dependant on the weather, and can increase or decrease by 20\% to 25\%.

\mbox{}

In a very simplified setting, assume three possible cases: good year (for every plant, the production is 20\% higher), average year, and bad year (for every plant, the production is 20\% lower).
The prices do not change.
\end{frame}

\begin{frame}
\frametitle{The farmer problem: scenario solutions}

{\blue New optimal solutions?}

\mbox{}

{\red Good year}. Total profit: \$167667.
\begin{center}
\begin{tabular}{lccc}
\hline
Culture & Wheat & Corn & Sugar beets \\
\hline
Surface (acres) & 183.33 & 66.67 & 250 \\
Production (T) & 550 & 240 & 6000 \\
Sales (T) & 350 & - & 6000 \\
Purchases (T) & - & - & - \\
\hline
\end{tabular}
\end{center}

\mbox{}

\mbox{}

{\red Bad year}. Total profit: \$59950.
\begin{center}
\begin{tabular}{lccc}
\hline
Culture & Wheat & Corn & Sugar beets \\
\hline
Surface (acres) & 100 & 25 & 375 \\
Production (T) & 200 & 60 & 6000 \\
Sales (T) & - & - & 6000 \\
Purchases (T) & - & 180 & - \\
\hline
\end{tabular}
\end{center}

\end{frame}

\begin{frame}
\frametitle{The farmer problem: scenarios}

The decisions considerably change with the weather conditions, but how to know them when deciding what to plant?
\mbox{}

The decisions $(x_1, x_2, x_3)$ have to be made now, but sales and purchases $(w_i, i=1,\ldots,4, y_j, j=1,2)$ depend on yields.

\mbox{}

{\blue Scenarios}.\\
Index $s = 1, 2, 3$, designing yields higher than the average, equal to the average, and lower than the average, respectively.\\
New variables $w_{is}$ and $y_{is}$.

\end{frame}

\begin{frame}
\frametitle{The farmer problem: extensive form}

We now want to maximize the {\red expected profit}.
Assuming that the 3 scenarios are equiprobable, we can form the new program
\begin{small}
\begin{align*}
\min\ & 150x_1 + 230x_2 + 260x_3 +\\
&  + \sum_{s = 1}^3 \frac{1}{3}(238y_{1s} - 170w_{1s} + 210y_{2s} - 150w_{2s} -
36w_{3s} - 10w_{4s}) \\
\text{t.q. } & x_1 + x_2 + x_3 \leq 500; \\
& 3x_1 + y_{11} - w_{11} \geq 200; 2.5x_1 + y_{12} - w_{12} \geq 200;
 2x_1 + y_{13} - w_{13} \geq 200; \\
& 3.6x_2 + y_{21} - w_{21} \geq 240; 3x_2 + y_{22} - w_{22} \geq
240;\\
& \quad{} 2.4x_2 + y_{23} - w_{23} \geq 240; \\
& w_{31} + w_{41} \leq 24 x_3; w_{32} + w_{42} \leq 20 x_3; w_{33} +
w_{43} \leq 16 x_3; \\
& w_{31} \leq 6000; w_{32} \leq 6000; w_{33} \leq 6000; \\
& x, y, w \geq 0.
\end{align*}
\end{small}
$\rightarrow$ {\blue extensive form}.

\end{frame}

\begin{frame}
\frametitle{The farmer problem: stages}

The seeding decisions are called {\red first-stage decisions}, while the sale and purchase decisions are called {\red second-stage decisions}.

\mbox{}

Total profit: \$108390.
\begin{center}
\begin{tabular}{llccc}
\hline
& Culture & Wheat & Corn & Sugar beets \\
\hline
First stage & Surface (acres) & 170 & 80 & 250 \\
\hline
$s = 1$ & Productions (T) & 510 & 288 & 6000 \\
& Sales (T) & 310 & 48 & 6000 \\
& Purchases (T) & - & - & - \\
\hline
$s = 2$ & Productions (T) & 425 & 240 & 5000 \\
& Sales (T) & 225 & - & 5000 \\
& Purchases (T) & - & - & - \\
\hline
$s = 3$ & Productions (T) & 340 & 192 & 4000 \\
& Sales (T) & 140 & - & 4000 \\
& Purchases (T) & - & 48 & - \\
\hline
\end{tabular}
\end{center}

\end{frame}

\begin{frame}
\frametitle{Observations}

The optimal decision has changed!!!

\mbox{}

Decision under {\red perfect information}:
if the farmer could know the scenario in advance, or wait to observe the realization of the random variables ({\red wait-and-see} approach),
the average annual profit would be \$115406.
The difference with the optimal decision under uncertainty is called {\blue expected value of perfect information (EVPI)}: profit loss due to uncertainty. 

\end{frame}

\begin{frame}
\frametitle{Observations: value of the stochastic solution}

\begin{itemize}
\item 
If the farmer only uses the average information, i.e. he replaces the random variables (r.v.) by their expectations, the average profit would be \$107240.
\item
Replacing the r.v. by their expectation 
leads to the expected value (EV) problem, delivering the expected value solution. \item
Here, the expectation of scenarios is the average year, but in general, the expectation will not necessarily correspond to a pre-existent scenario.
\item
The expectation of the expected value (EEV) problem is obtained by computing the expected profit over the scenarios when the expected value solution is always used.
\item
Here, it leads to a loss of \$1150 with respect to the solution of the stochastic problem. This difference is known as {\red value of the stochastic solution (VSS)}.
\end{itemize}

\end{frame}


\begin{frame}
\frametitle{Example: the newsvendor problem}

Source: Birge and Louveaux, Section~1.1.

\mbox{}

\begin{itemize}
	\item
	A newsvendor has to decide how many newspapers to buy in order to maximize his profit.
	However he does not know in advance how many newspapers he will be able to sell during a day (the demand).
	\item
	Each newspaper costs $c$, and can be sold at a price $q$.
	\item
	The newsvendor can turn back the unsold newspapers at the end of the day, and obtain a price $r$ for each of them
	\item
	Knowing the probability distribution $F(t) = P(\bomega \leq t$), how many newspapers should the newsvendor buy in order to maximize his revenue?
\end{itemize}

\end{frame}

\begin{frame}
\frametitle{The newsvendor problem (cont'd)}

\begin{itemize}
\item
With the previous definitions, the newsvendor would like to solve the following optimization problem:
\[
\max_{x \geq 0} -cx + \mathcal{Q}(x),
\]
\item
$\mathcal{Q}(x)$ is the expected sale amount if the newsvendor buy $x$ newspapers:
\[
\mathcal{Q}(x) = E_{\bomega} [ Q(x, \bomega)].
\]
%\item
%Dans le cadre linéaire, nous écrirons aussi souvent
%$Q(x, \omega) = v(h(\omega) - T (\omega)x)$.
\item
Here $Q(x, \omega)$ is the amount of money obtained by the newsvendor if he buys $x$ newspaper and the demand is $\omega$.
\end{itemize}

\end{frame}

\begin{frame}
\frametitle{The newsvendor problem (cont'd)}

\begin{itemize}
\item
As previously, we could construct an equivalent linear problem (presented in Birge and Louveaux).
\item
Can we simplify? It is easy to see that
\[
Q(x, \omega) =
\begin{cases} 
qx & \mbox{if } x \leq \omega, \\
q\omega + r(x - \omega) & \mbox{if } x \geq \omega.
\end{cases}
\]
Therefore,
\begin{align*}
\mathcal{Q}(x) & = E_{\bomega} [ Q(x, \bomega) ] =
\int_{-\infty}^{\infty} Q(x, \omega) dF(\omega) 
\\ & = \int_{-\infty}^x(q\omega + r(x - \omega))dF(\omega) + \int^{\infty}_x
qxdF(\omega).
\end{align*}
\end{itemize}

\end{frame}

\begin{frame}
\frametitle{The newsvendor problem (cont'd)}

Therefore, we have
\begin{align*}
\mathcal{Q}(x) & = (q-r) \int_{-\infty}^x \omega dF(\omega) + rx
\int_{-\infty}^x dF(\omega) + qx \int_x^{\infty} dF(\omega) \\
& = (q-r) \int_{-\infty}^x \omega dF(\omega) + rx F(x) + qx (1-F(x)) \\
& = (q-r) \left[ \int_{-\infty}^x \omega dF(\omega) - x F(x) \right] + qx. \\
\end{align*}

\end{frame}

\begin{frame}
\frametitle{Integration by parts}

Assume that $F$ satisfies $\lim_{t \rightarrow -\infty}
tF(t) = 0$.

\mbox{}

We can then integrate by parts to obtain:
\begin{align*}
\int_{-\infty}^x \omega dF(\omega) &= \omega F(\omega) |_{-\infty}^x -
\int_{-\infty}^x F(\omega) d\omega \\
&=  xF(x) - \int_{-\infty}^x F(\omega) d\omega.
\end{align*}

Thus,
\[
\mathcal{Q}(x) = qx - (q - r)\int_{-\infty}^x F(\omega)d\omega.
\]
%Par conséquent, nous allons réviser ce point,\ldots lors du prochain
%épisode de IFT6512!

\end{frame}

\begin{frame}
\frametitle{Solution of the second stage}

Recall the initial problem\ldots
\[
\max_{x \geq 0} -cx + \mathcal{Q}(x),
\]
We have to solve this problem.
We will consider the associated optimality conditions.

\mbox{}

Assuming $x \ne 0$, the solution of the second-stage is obtained by computing the zero of the objective gradient.
As
\[
\frac{d}{dx} \mathcal{Q}(x) = q - (q - r)F(x),
\]
we have
\[
-c + q - (q - r)F(x) = 0
\]

\end{frame}

\begin{frame}[fragile]
\frametitle{The newsvendor problem (cont'd)}

The solution $x^*$ is therefore
\[
x^* = F^{-1} \left( \frac{q-c}{q-r} \right).
\]

\mbox{}

Example: $c = 0.15$, $q = 0.25$, $r = 0.02$, $\omega \sim N(650, 80^2)$. Alors
\[
x^* = N^{-1}_{(650,80^2)}(0.1/0.23).
\]
Since $N(650, 80^2) \sim 80\Phi+650$, where $\Phi$ is the distribution function of a $N(0,1)$, it easy to show that
\[
x^* = 80\Phi^{-1}(0.1/0.23) + 650 \approx 636.86.
\]
In Julia, we can compute this value as
\begin{verbatim}
using Distributions
d = Normal(650,80)
quantile(d, 0.1/0.23)
\end{verbatim}

\end{frame}

\begin{frame}
\frametitle{Marginal revenue}

{\sl Other interpretation}, more intuitive: assume that the vendor has bought $t$ journaux.
What is the expected marginal revenue if he buys an additional newspaper?
On an economical point of view, we would like this marginal revenue to be equal to 0.

\mbox{}

The expected marginal revenue (MR) is
\begin{align*}
MR(t) &= -c + qP[\bomega \geq t] + rP[\bomega \leq t]\\
&= -c + q(1-F(t)) + rF(t).
\end{align*}
If we set the marginal revenue to 0, we get
\[
MR(t) = 0 \mbox{ iff } F(t) = \frac{q-c}{q-r},
\]
and we recover the previous solution.

%Pour des compléments, consultez Linderoth.

\end{frame}

%%%%%%%%%%%%%
\begin{frame}
\frametitle{Formulation with recourse}

More generally, we consider the (linear) program
\begin{align*}
\min\ & c^Tx+E_{\bomega}[q(\bomega)^Ty(\bomega)] \\
\mbox{t.q. } & Ax = b, \\
& T(\omega)x + Wy(\omega) = h(\omega),\ \forall \omega
\in \Omega \\
& x \in X, \\
& y(\omega) \in Y,\ \forall \omega.
\end{align*}

\mbox{}

Fixed recourse: $W$ does not change with the scenario.

\mbox{}

How to decide over $y$?

\end{frame}

\begin{frame}
\frametitle{Some definitions}

\[
\min_{x \in X | Ax = b} \left\lbrace c^Tx + E_{\bomega} \left[ \min_{y
      \in Y} q(\bomega)^T y | W y = h(\bomega) - T(\bomega)x \right] \right\rbrace.
\]
\begin{itemize}
\item
{\red Second stage function}, or {\red recourse function} $v :
\Omega \times \rit^m \rightarrow \rit$:
\[
v(\omega, z) \overset{def}{=} \min_{y \in Y} \lbrace q(\omega)^Ty | W y = z \rbrace;
\]
%cette fonction décrit les coûts correspondants à n'importe quel
%vecteur $z$ de ``déviations des contraintes aléatoires $T(\omega)x =
%f(\omega)$.''
\item
{\red Expected value function}, or recourse of minimum expectation
$\mathcal{Q} : \rit^n \rightarrow \rit$:
\[
\cQ(x) = E_{\bomega} [ v (\bomega, h(\bomega) - T(\bomega) x)].
\]
It describes the expected recourse cost, for any first-stage decision $x \in \rit^n$.
\end{itemize}

\end{frame}

\begin{frame}
\frametitle{The two-stage (linear) stochastic program}

One can reformulate our optimization problem as
\[
\min_{x \in X} \left\lbrace c^T x + \mathcal{Q}(x) \ |\  Ax = b \right\rbrace.
\]
It is a (nonlinear) optimization problem in $\rit^n$.

\mbox{}

In terms of $y$'s:
\begin{align*}
\min_{x,\ y(\bomega)}\ & E_{\bomega} [ c^T x + q(\bomega)^T y(\bomega) ] \\
\mbox{s.t. } & Ax = b & \mbox{first-stage constraints} \\
& T(\omega)x + Wy(\omega) = h(\omega),\ \forall \omega \in \Omega &
\mbox{second-stage constraints} \\
& x \in X,\ y(\omega) \in Y.
\end{align*}

Consider the (discrete) case where $\Omega = \lbrace \omega_1,\
\omega_2,\ldots,\omega_S \rbrace \subset \rit^r$.
\begin{align*}
P(\omega = \omega_s) & = p_s,\ s = 1, 2,\ldots . . . , S \\
T_s & = T (\omega),\ h_s = h(\omega) 
\end{align*}

\end{frame}

\begin{frame}
\frametitle{Deterministic equivalent}

\begin{align*}
\min_{x, y_1, \ldots, y_S}\ & c^T x + p_1 q_1^T y_1 + p_2 q_2^T y_2 + \ldots
p_S q_S^Ty_S \\
\mbox{t.q. } & \\
& \begin{matrix} Ax & & & & & = b\\
T_1 x & + W y_1 & & & & = h_1 \\
T_2 x & & + W y_2 & & & = h_2 \\
\vdots & & & \ddots & \\
T_S x & & & & + W y_s & = h_s
\end{matrix} \\
& x \in X,\ y_1 \in Y,\ y_2 \in Y,\ldots, y_s \in Y.
\end{align*}

\end{frame}

\begin{frame}
\frametitle{Deterministic equivalent (II)}

\begin{itemize}
\item
$y_s = y(\omega_s)$ is the recourse action to take if the scenario $\omega_s$ occurs.
\item
{\color{blue}\it Advantage}: it is a linear program.
\item
{\color{blue}\it Disadvantage}: it is a linear program of (very) high dimension:
\begin{itemize}
\item
$n+pS$ variables;
\item
$m_1+mS$ constraints.
\end{itemize}
But the constraints matrix has a staircase structure.\\
It is possible to exploit it (L-Shaped algorithm -- Benders decomposition).
\end{itemize}

\end{frame}

\begin{frame}
\frametitle{Large scale,\ldots and?}

Assume that we have $r$ random variables ($\Omega \subset
\rit^r$).
\begin{itemize}
\item
Consider the following problem (source: Linderoth).
A Telecom company want to expand its network in order to meet an unknown (random) demand.
\item
There are 86 unknown demands.
Each demand is independant and take a value in a set of 7 values.
Consequently
\[
S = |\Omega| = 7^{86} \approx 4.77\times10^{72}.
\]
\ldots number of subatomic particles in the universe!
\item
It can be even worse\ldots\\
If $\Omega$ is not finite, but holds an infinite number of elements?
It is especially true with continuous random variables.
Our ``deterministic equivalent'' would have an infinite number of variables and constraints!
\item
We can solve an approximate problem, obtained by sampling over the random vector.
\end{itemize}
\end{frame}

\begin{frame}
\frametitle{Decomposition methods}

{\red General principle}: the nonlinear term in the objective, that is the recourse function $\cQ(x)$, requires to solve all the linear second-stage programs.

\mbox{}

Is it possible to avoid the repeated second-stage functions evaluations?

\mbox{}

{\blue Idea}: build a master problem in $x$, but compute the complete objective function (involving first- and second-stage decision) only as a subproblem.

\end{frame}

\end{document}
